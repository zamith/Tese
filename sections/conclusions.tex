\label{sec:conclusion}
The differences between a generic distributed database and RDBMS, have been shown, as well as how they are translated into actual implementations of such models, in the examples of Apache Cassandra and Apache DerbyDB.

Also, through a proof of concept, it has been demonstrated that joining these two different worlds is, indeed, possible, using the \ac{rdbms} as an \ac{sql}  interpreter, responsible for asking for the necessary keys, and the NoSQL system as the actual place where data is stored.

The actual code connecting the two will have to be studied with greater depth and the best approach to be taken will also have to be very well thought. This, as well as the rewriting of the code, are the future work, regarding this problem.

Other relevant work in the area has also been referred to, as is the case of \ac{cql}, which development will be attentively followed.

Being that NoSQL is a rapidly changing area, so is it's state of the art, meaning that the statements and assumptions made, refer to the knowledge and expertise available at the time of writing.