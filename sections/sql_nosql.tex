\section{SQL Database Management System}
\label{sec:rdbms}

\begin{quote}
	A method for structuring data in the form of sets of records or tuples so that relations between different entities and attributes can be used for data access and transformation. 
\end{quote} 
\begin{flushright}Burroughs, 1986\end{flushright}
	
This work focuses on the \ac{sql} kind of \ac{dbms} since it is the most widely used and, therefore the one used for the proof of concept.

\subsection{Concept}
A relational database is a database that is perceived by the user as a collection of two-dimensional tables that are manipulated a set at a time, instead of a record at a time.  

It has a rigorous design methodology that is achieved through normalization\footnote{ the process of organizing data to minimize redundancy.}. Moreover it is easily modifiable by adding new tables and rows, even though the schema is rigid, i.e. all the rows in a table must have the same columns. 

One of the main advantages of this approach is the very powerful and flexible join mechanism, based on algebraic set\footnote{ group of common elements where each member has some unique aspect or attribute} theory, providing fast responses to complex queries.

Since its appearance a lot of work has been done in order to fasten the processing, from multi-threaded or parallel servers to the usage of indexes\footnote{ used to find rows with specific column values fast at the cost of slower writes and increased storage space.} and fast networks. 

\subsection{Components}
Every \ac{dbms}  has four common components, its building blocks. They may vary from one system to another, but the general purpose of each of these components is always the same.

\subsubsection{Modeling language}
First of all, there is the modeling language, that defines the schema of the database, that is, the way it is structured. These models range from the hierarchical, to the network, object, multidimensional and to the relational, that can be combined to provide an optimal system. The most commonly used is the relational structure, that uses two-dimensional rows and columns to store data, forming records, that can be connected to each other by key values. 

In order to get more practical and faster systems, the most used model today is actual a relational model embedded with \ac{sql} .

\subsubsection{Data Structure}
Every database has its own data structures (fields, records, files and objects) optimized to deal with very large amounts of data stored on a permanent data storage device (which is obviously slow, when compared to volatile memory).

\subsubsection{Database Query Language}
A database query language allows users to interrogate the database, analyze and update data, and control its security. Users can be granted different types of privileges, and the identity of said users is guaranteed using a password. The most widely used language nowadays is \ac{sql} , which provides the user with four main operations, know as CRUD (Create, Read, Update, Delete).  

\subsubsection{Transactions}
A \ac{rdbms} must provide a transactional system with ACID properties.

\section{NoSQL}
\label{sec:nosql}
Relational databases are not tuned for certain data intensive applications, as serving pages on high traffic websites or streaming media, therefore show poor performance in these cases. Usually they are tuned either for small but frequent read/write transactions or for large batch transactions, used mostly for reading purposes. On the other hand, NoSQL addresses services that have heavy read/write workloads, as the Facebook's inbox search \cite{lakshmanMalik}.

As stated earlier, NoSQL often provides weak consistency guarantees, as eventual consistency \cite{Vogels2008}, and many of these systems employ a distributed architecture, storing the data in a replicated manner, often using a distributed hash table \cite{Tanner}. This allows for the system to scale out with the addition of new nodes and to tolerate failure of a server.

\subsection{Taxonomy}
NoSQL implementations can be categorized according to the way they are implemented, being that they are a document store, a key/value store on disk or a cache in \ac{ram}, a tuple store, an object database, or as the one used in this work, an eventually-consistent key/value store.\\

\section{SQL vs NoSQL}
Databases that use SQL as their query language and the ones that use a low level, one record at a time interface (most NoSQL implementations) can coexist, and they have, since they both have good reasons and scenarios for being used. This is a topic that has been introduced in chapter \ref{chap:intro} and will be further discussed in this section.

There are two main reasons for moving from \acp{rdbms} to this alternate \ac{dbms} technology, that are performance and flexibility. 

Performance issues in a \ac{rdbms} like MySQL come when you have too much data and/or requests per second for its capabilities, leaving you with three options, to shard the data and partition it, to pay big license fees for an enterprise \ac{rdbms}\footnote{in some cases even this is not feasible}, or move to something other that a \ac{rdbms}.

Flexibility issues arise when your data does not conform to the rigidness of the relational schema, which means you a have the need for something more flexible.

According to Michael Stonebraker's blog post~\cite{stoneDisc}, improving performance on a \ac{dbms} depends on removing overhead.\emph{``Such overhead has nothing to do with \ac{sql}, but instead revolves around traditional implementations of ACID transactions, multi-threading, and disk management. To go wildly faster, one must remove all four sources of overhead, discussed above. This is possible in either a SQL context or some other context.''}

The four sources of overhead mentioned are:
\begin{description}
	\item[Logging] Every update must be written to log an pushed to disk 
	\item[Locking] Before accessing a record, a transaction must set a lock on it
	\item[Latching] Updates to shared data structures must be done in a multi-threaded environment
	\item[Buffer Management] A buffer pool manages which set of disk pages is cached in memory at any given time
\end{description}

However, he also states that \emph{``the single-node performance of a NoSQL, disk-based, non-ACID, multithreaded system is limited to be a modest factor faster than a well-designed stored-procedure SQL OLTP engine''}. Meaning that a NoSQL solution performance can in some cases be reached but only with systems designed by experts and with a high degree of complexity, unlike the NoSQL solution which is fairly simple to set up. This is particularly true in extremely large \acp{rdbms} implementations, such as Facebook's, that uses MySQL with memcached and non-transparent sharding\footnote{``transparent sharding (is) a data management strategy in which data is assigned to multiple servers (or CPUs, cores, etc.), yet looks to programmers and applications as if it were managed by just one.'' - \url{http://www.dbms2.com/2011/02/24/transparent-sharding/} (27/09/2011)}. \\

According to yet another blog post by Stonebraker~\cite{stoneEnter}, 61\% of enterprise users are either ignorant about or uninterested in NoSQL. This happens mainly due to three reasons, because it \textbf{does not provide ACID}, it has a \textbf{low level interface} instead of a high-level language as \ac{sql} and because there is \textbf{no standard} for NoSQL interfaces. 

Our proposed solution covers all of these three problems, at the cost of having more two more of the overhead sources than a typical NoSQL implementation, which has only the overhead of latching and buffer management.

\section{Case Study}
To accomplish the goals proposed in the introduction, there was a need to choose one \ac{rdbms} and one NoSQL implementation in order to develop a solution as both a proof of concept and a way to get real results. The chosen systems were the Apache Derby as the database manager and the Apache Cassandra as the NoSQL implementation for the following reasons:

\begin{itemize}
	\item Written in Java
	\item Open-source
	\item Previous experience with both
\end{itemize}