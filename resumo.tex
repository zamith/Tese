
\paragraph{}

\todo[inline]{Melhorar tradução}

Nos últimos anos houve um enorme crescimento na área das bases de dados distribuídas, especialmente com o movimento NoSQL. Estas bases de dados têm  como propósito não ter \emph{schema} nem ser tão rígidas, como as suas \emph{counterparts} relacionais, no que toca ao modelo de dados, por forma a atingir uma maior escalabilidade.

A sua \emph{API de queries} tem tendêcia a ser bastante reduzida e simples (normalmente uma operação para inserir, uma para ler e outra para remover dados), o que lhes permite ter leituras e escritas muito rápidas. Todas estas propriedades podem também ser vistas como uma perda de capacidade tanto a nível de coerência como de poder de \emph{query}. Assim, houve a necessidade de expor os vários argumentos contra e a favor destas propriedades, bem como as tentativas que já foram e estão a ser feitas para aproximar estas duas tecnologias e o porquê de não serem satisfatórias.

Neste contexto, o trabalho apresentado nesta dissertação de mestrado é o resultado da avaliação de como integrar propriedades de bases de dados relacionais e não relacionais. A solução proposta usa o Apache Derby DB, o Apache Cassandra e o Apache Zookeeper, tendo benefícios e \emph{drawbacks} que foram identificados e analisados.

Finalmente, para avaliar a solução proposta e implementada, usámos um \emph{workload} baseado naquele definido pela \emph{benchmark} TPC-W, largamente utilizada em servidores web transaccionais orientados para o negócio.Segundo esta especificação avaliámos a nossa proposta em diferentes cenários e configurações.

