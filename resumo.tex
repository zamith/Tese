
\paragraph{}


Existe nos dias de hoje uma necessidade crescente da utilização de replicação em bases de dados, sendo que a construção de aplicações de alta performance, disponibilidade e em grande escala dependem desta para manter os dados sincronizados entre servidores e para obter tolerância a faltas.


%A necessidade da utilização de replicação em bases de dados é cada vez maior nos dias de hoje sendo que, a construcção de aplicações de alta performance, disponibilidade e em grande escala dependem desta para manter os dados sincronizados entre servidores e para obter tolerância a faltas.

%There is nowadays an increasing need for database replication, as the construction of high performance, highly available, and large-scale applications depends on it to maintain data synchronized across multiple servers and to achieve fault tolerance.

%A replicação é uma técnica essencial para sistemas de grande escala, alta performance e disponibilidade. A construcção destes sistemas depende da replicação para resolver o problema de manter os dados sincronizados entre servidores e para obter tolerância a faltas.

Uma abordagem particularmente popular, é o sistema código aberto de gestão de bases de dados MySQL e seu mecanismo interno de replicação assíncrona. As limitações impostas pelo MySQL nas topologias de replicação significam que os dados tem que passar por uma série de saltos ou que cada servidor tem de lidar com um grande número de réplicas. Isto é particularmente preocupante quando as actualizações são aceites por várias réplicas e em sistemas de grande escala. Observando as topologias mais comuns e tendo em conta a assincronia referida, surge um problema, o da frescura dos dados. Ou seja, o facto das réplicas não possuírem imediatamente os dados escritos mais recentemente. Este problema vai de encontro ao estado da arte em comunicação em grupo. %tendo em conta as características inerentes a este. Garantias como confiança, ordem, estabilidade, e entrega de mensagens.

Neste contexto, o trabalho apresentado nesta dissertação de Mestrado resulta de uma avaliação dos modelos e mecanismos de comunicação em grupo, assim como as vantagens práticas da replicação baseada nestes. A solução proposta estende a ferramenta MySQL Proxy com plugins aliados ao sistema de comunicação em grupo Spread oferecendo a possibilidade de realizar, de forma transparente, replicação activa e passiva.

Finalmente, para avaliar a solução proposta e implementada utilizamos o modelo de carga de referência definido pelo TPC-C, largamente utilizado para medir o desempenho de bases de dados comerciais. Sob essa especificação, avaliamos assim a nossa proposta em diferentes cenários e configurações.  % As premissas de um melhor desempenho em comparação com o mecanismo de replicação tradicional do MySQL são confirmadas pelos resultados de desempenho.


